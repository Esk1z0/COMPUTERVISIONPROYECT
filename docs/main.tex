\documentclass[a4paper, 11pt]{article}
\usepackage[left=2.5cm,right=2.5cm,top=3.5cm,bottom=2.5cm]{geometry} % Set paper margins
\usepackage{graphicx} % Required for inserting images
\usepackage{url}
\usepackage{xurl}
\usepackage[spanish,es-tabla]{babel}
\usepackage{fancyhdr} % Head/foot config
\usepackage{pdfpages}
\usepackage{marvosym}
\usepackage{listings}
\usepackage{tabularx,ragged2e}
\newcolumntype{C}{>{\Centering\arraybackslash}X} % centered "X" column
\usepackage{amsmath}
\emergencystretch 5em%
\usepackage{float}
\usepackage{booktabs}
\usepackage{placeins}

\renewcommand{\lstlistingname}{Listado}
\lstset{
  basicstyle=\ttfamily,
  columns=fullflexible,
  keepspaces=true,
  captionpos=b,
  aboveskip=12pt
}

\newcommand{\leftalign}[1]{
  \setlength{\parindent}{0pt}
  \setlength{\hangindent}{\linewidth}
  \hangafter=1 #1
}

\begin{document}

%%% PORTADA
\begin{center}

\thispagestyle{empty}
\LARGE{Universidad Politécnica de Madrid}\\[2ex]
\large{Máster Universitario en Inteligencia Artificial}\\[2ex]
\vspace{0.3cm}

\begin{center}

\includegraphics[width=10cm]{figures/logo.png}\\
\vspace{0.9cm}
\textbf{\LARGE{Proyecto ffNN}}
\bigskip\bigskip\bigskip\par
\textbf{Nombre1}\\
Universidad Politécnica de Madrid\\
Madrid\\
\url{email1@alumnos.upm.es}
\vspace{1.2cm}\par
\textbf{Yeray Martínez Martínez}\\
Universidad Politécnica de Madrid\\
Madrid\\
\url{y.mmartinez@alumnos.upm.es}
\vspace{1.2cm}\par
\textbf{Nombre3}\\
Universidad Politécnica de Madrid\\
Madrid\\
\url{email3@alumnos.upm.es}

\end{center}

\end{center}

% Configuración de headers y footers
\pagestyle{fancy}
\fancyhf{}
\fancyhead[R]{\textit{FFNN PROJECT}}
\fancyfoot[R]{\thepage}

%%% RESUMEN (pag 2)

\pagebreak
\pagenumbering{roman}

\renewcommand{\abstractname}{Resumen}
\begin{abstract}
\setlength{\parskip}{\baselineskip}
Resumen.
\end{abstract}

\pagebreak

%%% INDICE (pag 3)

% Contenidos
\renewcommand{\contentsname}{Índice de contenidos}
\tableofcontents
\pagebreak
% Tablas
\renewcommand{\listtablename}{Índice de tablas}
\listoftables
\pagebreak
% Figuras
\renewcommand{\listfigurename}{Índice de figuras}
\listoffigures
\pagebreak
% Listados
\renewcommand{\lstlistlistingname}{Índice de listados}
\lstlistoflistings
\pagebreak

\pagebreak

%%% CUERPO

\pagenumbering{arabic}
\setlength{\parskip}{\baselineskip}

% Introducción
\input{intro}
\pagebreak

% Procesamiento de datos
\section{Procesamiento de datos}

Texto.
\pagebreak

% Desarrollo
\section{Desarrollo}

\subsection{Subsección 1}

Texto.


\FloatBarrier

\subsection{Subsección 2}

Texto.


\FloatBarrier
\pagebreak

% Conclusiones
\input{conclusions}
\pagebreak

%%% BIBLIOGRAFIA

\bibliographystyle{IEEEtran}
\nocite{*}
\renewcommand{\refname}{Bibliografía}
\bibliography{bibliography}

\end{document}
